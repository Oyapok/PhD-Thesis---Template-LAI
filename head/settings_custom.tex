%%%%%%%%%%%%%%%%%%%%%%%%%%%%%%%%%%%%%%%%%%%%%%
%
%		Thesis Settings
%		Custom settings
%
%		2011
%
%%%%%%%%%%%%%%%%%%%%%%%%%%%%%%%%%%%%%%%%%%%%%%

% %
% %   Use this file for your own custom packages, command-definitions, etc...
% %
% % 
% % Packages for references - cleverref must be last
% \usepackage{nameref}
% \usepackage{hyperref}
% \usepackage{cleveref}
% \usepackage[shortlabels]{enumitem}
% % Reduce spacing in bibliography
% \setlength{\bibsep}{0pt plus 0.3ex}
% % Allow equations to break between pages
% \allowdisplaybreaks
% % Penalty for widow and orphan
% \widowpenalty=9999
% \clubpenalty=9999
% %Penalty for relation and binary operation breaks in equations
% \relpenalty=9999
% \binoppenalty=9999

\usepackage{algorithmic}
\usepackage{float}
\usepackage{textcomp}

\usepackage{siunitx} % To use SI units in text. Use \SI{<value>}{<unit>}
\usepackage[export]{adjustbox}
\usepackage{IEEEtrantools}
\usepackage{adjustbox}  % To have access to macro that allow to adjust boxed contents
\usepackage{makecell}
\usepackage{wrapfig}

\usepackage{trfsigns} % Transformation Symbol o---o \laplace and \Laplace
\newcommand{\vlaplace}[1][]{\mbox{\setlength{\unitlength}{0.1em}%
                            \begin{picture}(10,20)%
                              \put(3,2){\circle{4}}%
                              \put(3,4){\line(0,1){12}}%
                              \put(3,18){\circle*{4}}%
                              \put(10,7){#1}
                            \end{picture}%
                           }%
                     }%

\newcommand{\vLaplace}[1][]{\mbox{\setlength{\unitlength}{0.1em}%
                            \begin{picture}(10,20)%
                              \put(3,2){\circle*{4}}%
                              \put(3,4){\line(0,1){12}}%
                              \put(3,18){\circle{4}}%
                              \put(10,7){#1}
                            \end{picture}%
                           }%
                     }%

% TIKZ packages and commands to draw pictures, on pictures, and electrical circuits
% \usepackage{tikz}
\usepackage[outline]{contour} % To have a colored border on text

\usetikzlibrary{babel}
\usetikzlibrary{shapes, arrows}
\usetikzlibrary{patterns,decorations.pathreplacing}
\usetikzlibrary{patterns.meta}
\usetikzlibrary{shapes.arrows}
\usetikzlibrary{shapes.multipart}
\usetikzlibrary{spy}
\usetikzlibrary{calc}
\usepackage{marvosym}
\tikzset{>=latex}

\usepackage{pgfplots}        % make nice plots using raw data
\pgfplotsset{compat=1.18}
% \usepgfplotslibrary{external}
% \usetikzlibrary{external}
% \tikzexternalize[prefix=tikz/]            % To reduce compilation time by reusing a once compiled pgf plot
\pgfkeys{/pgf/number format/.cd,1000 sep={\,}}

\newcommand{\tikzFiguresScale}{0.8}

% New command to create the "kinky line" or line jumping over the other in electrical circuits
\newcommand*\circled[1]{\tikz[baseline=(char.base)]{
    \node[shape=circle,draw,inner sep=1pt] (char) {#1};}}
\usepackage[siunitx,europeanresistors,EFvoltages]{circuitikz} % To have the european convention for the drawing of electrical components. Also to have the current and voltage arrows going from high to low potential
\ctikzset{voltage/bump b=18pt,voltage/european label distance=15pt,voltage/distance from node=.1}
\tikzset{% Allows to have lines in a circuit that jump over one another (kinky cross)
  declare function={% in case of CVS which switches the arguments of atan2
    atan3(\a,\b)=ifthenelse(atan2(0,1)==90, atan2(\a,\b), atan2(\b,\a));},
  kinky cross radius/.initial=+.125cm,
  @kinky cross/.initial=+, kinky crosses/.is choice,
  kinky crosses/left/.style={@kinky cross=-},kinky crosses/right/.style={@kinky cross=+},
  kinky cross/.style args={(#1)--(#2)}{
    to path={
      let \p{@kc@}=($(\tikztotarget)-(\tikztostart)$),
          \n{@kc@}={atan3(\p{@kc@})+180} in
      -- ($(intersection of \tikztostart--{\tikztotarget} and #1--#2)!%
             \pgfkeysvalueof{/tikz/kinky cross radius}!(\tikztostart)$)
      arc [ radius     =\pgfkeysvalueof{/tikz/kinky cross radius},
            start angle=\n{@kc@},
            delta angle=\pgfkeysvalueof{/tikz/@kinky cross}180 ]
      -- (\tikztotarget)}}}

      
%-------------------
% Various arrow styles
%-------------------
\usetikzlibrary{arrows}
\tikzset{
  double -latex/.style args={#1 colored by #2 and #3}{    
    -latex,line width=#1,#2,
    postaction={draw,-latex,#3,line width=(#1)/3,shorten <=(#1)/4,shorten >=4.5*(#1)/3},
  },
  double round cap-latex/.style args={#1 colored by #2 and #3}{    
    round cap-latex,line width=#1,#2,
    postaction={draw,round cap-latex,#3,line width=(#1)/3,shorten <=(#1)/4,shorten >=4.5*(#1)/3},
  },
  double round cap-stealth/.style args={#1 colored by #2 and #3}{
    round cap-stealth,line width=#1,#2,
    postaction={round cap-stealth,draw,,#3,line width=(#1)/3,shorten <=(#1)/3,shorten >=2*(#1)/3},
  },
  double -stealth/.style args={#1 colored by #2 and #3}{
    -stealth,line width=#1,#2,
    postaction={-stealth,draw,,#3,line width=(#1)/3,shorten <=(#1)/3,shorten >=2*(#1)/3},
  },
}
      
%-------------------
% My colors
%-------------------
\definecolor{superlightestblue}{rgb}{0.92,0.97,1}
\definecolor{lightestblue}{rgb}{0.88,0.93,1}
\definecolor{lightblue}{rgb}{0.72,0.77,0.85}
\definecolor{superlightergray}{rgb}{0.85, 0.85, 0.85}
\definecolor{superlightgray}{rgb}{0.7, 0.7, 0.7}
\definecolor{lightergray}{rgb}{0.5, 0.5, 0.5}
\definecolor{lightgray}{rgb}{0.35, 0.35, 0.35}
\definecolor{darkgray}{rgb}{0.25, 0.25, 0.25}
\definecolor{lightmauve}{rgb}{0.86, 0.82, 1.0}
\definecolor{lightpastelpurple}{rgb}{0.69, 0.61, 0.85}
\definecolor{mauve}{rgb}{0.88, 0.69, 1.0}
\definecolor{mygreen}{RGB}{0,204,0}
\definecolor{myaqua}{rgb}{0.0, 1.0, 1.0}
% Matlab colors
\definecolor{matlabblue}{rgb}{0.000,0.447,0.741}
\definecolor{matlabred}{rgb}{0.850,0.325,0.098}
\definecolor{matlaborange}{rgb}{0.929,0.694,0.125}
\definecolor{matlabpurple}{rgb}{0.494,0.184,0.556}
\definecolor{matlabgreen}{rgb}{0.466,0.674,0.188}
\definecolor{matlablightblue}{rgb}{0.301,0.745,0.933}
\definecolor{matlabdarkred}{rgb}{0.635,0.078,0.184}
% Matlab "Jet" 10 colors colormap
\definecolor{jetdarkblue}{rgb}{0.00000,0.00000,0.66667}%
\definecolor{jetblue}{rgb}{0.00000,0.00000,1.00000}
\definecolor{jetlightblue}{rgb}{0.00000,0.33333,1.00000}%
\definecolor{jetlighterblue}{rgb}{0.00000,0.66667,1.00000}%
\definecolor{jetturquoise}{rgb}{0.00000,1.00000,1.00000}%
\definecolor{jetteal}{rgb}{0.33333,1.00000,0.66667}%
\definecolor{jetgreen}{rgb}{0.66667,1.00000,0.33333}%
\definecolor{jetyellow}{rgb}{1.00000,1.00000,0.00000}%
\definecolor{jetorange}{rgb}{1.00000,0.66667,0.00000}%
\definecolor{jetred}{rgb}{1.00000,0.333333,0,00000}
% My Rainbow colors
\definecolor{rainbowyellow}{RGB}{255,240,0}
\definecolor{rainboworange}{RGB}{240,130,0}
\definecolor{rainbowred}{RGB}{205,40,30}
\definecolor{rainbowmagenta}{RGB}{230,60,165}
\definecolor{rainbowlightpurple}{RGB}{170,75,250}
\definecolor{rainbowpurple}{RGB}{100,30,150}
\definecolor{rainbowblue}{RGB}{0,100,240}
\definecolor{rainbowlightblue}{RGB}{0,210,230}
\definecolor{rainbowgreen}{RGB}{0,200,40}
\definecolor{rainbowlightgreen}{RGB}{180,240,0}
  
%---------------------------------
% To create boxes around equations
%---------------------------------
\usetikzlibrary{shadows} %defines shadows
\usepackage[framemethod=tikz]{mdframed}
\global\mdfdefinestyle{myboxstyle}{%
shadow=true,
linecolor=black,
shadowcolor=black,
shadowsize=6pt,
nobreak=false,
innertopmargin=10pt,
innerbottommargin=10pt,
leftmargin=5pt,
rightmargin=5pt,
needspace=1cm,
skipabove=10pt,
skipbelow=15pt,
middlelinewidth=1pt,
afterlastframe={\vspace{5pt}},
aftersingleframe={\vspace{5pt}},
tikzsetting={%
draw=black,
very thick}
}
% framed box that allows page-breaks
\newmdenv[style=myboxstyle]{whitebox}
\newmdenv[style=myboxstyle,backgroundcolor=black!20]{graybox}
% framed box that CANNOT be broken at end of page
\newmdenv[style=myboxstyle,nobreak=true]{blockwhitebox}
\newmdenv[style=myboxstyle,backgroundcolor=black!20,nobreak=true]{blockgraybox}
% invisible box that CANNOT be broken at end of page
\newmdenv[nobreak=true,hidealllines=true]{blockbox}
\usepackage{booktabs}
%--------------------------------------------

%---------------------------------
% To create animation in the pdf
%---------------------------------
\usepackage{animate}

% Examples:

% \begin{animateinline}[controls,autoplay,loop]{2}
% \multiframe{8}{n=1+1}{
%   \begin{tikzpicture}[scale=10,rotate=90]
%     \draw (-.1,-.2) rectangle (.4,0.2);
%     \draw [blue,opacity=0.5,line width=0.1cm,line cap=round]
%       l-system [l-system={A,axiom=A,order=\n,angle=45,step=0.25cm}];
%   \end{tikzpicture}    
% }
% \end{animateinline}

% \animategraphics[controls=none,width=1in,loop,autoplay,poster=first,scale=3]{12}{images/spongebob/spongebob-}{0}{27}
%--------------------------------------------

% For arrows styles
\tikzset{
  double -latex/.style args={#1 colored by #2 and #3}{    
    -latex,line width=#1,#2,
    postaction={draw,-latex,#3,line width=(#1)/3,shorten <=(#1)/4,shorten >=4.5*(#1)/3},
  },
  double round cap-latex/.style args={#1 colored by #2 and #3}{    
    round cap-latex,line width=#1,#2,
    postaction={draw,round cap-latex,#3,line width=(#1)/3,shorten <=(#1)/4,shorten >=4.5*(#1)/3},
  },
  double round cap-stealth/.style args={#1 colored by #2 and #3}{
    round cap-stealth,line width=#1,#2,
    postaction={round cap-stealth,draw,,#3,line width=(#1)/3,shorten <=(#1)/3,shorten >=2*(#1)/3},
  },
  double -stealth/.style args={#1 colored by #2 and #3}{
    -stealth,line width=#1,#2,
    postaction={-stealth,draw,,#3,line width=(#1)/3,shorten <=(#1)/3,shorten >=2*(#1)/3},
  },
}

% To annotate a figure
%%%%%%%%%%%%%%%%%%%%%%%%%%%%%%%%%%%%%%%%%%%%%%%%%%%%%%%%%%%%%%%%%%%%%%
% LaTeX Overlay Generator - Annotated Figures v0.0.1
% Created with http://ff.cx/latex-overlay-generator/
%%%%%%%%%%%%%%%%%%%%%%%%%%%%%%%%%%%%%%%%%%%%%%%%%%%%%%%%%%%%%%%%%%%%%%
%\annotatedFigureBoxCustom{bottom-left}{top-right}{label}{label-position}{box-color}{label-color}{border-color}{text-color}
\newcommand*\annotatedFigureBoxCustom[8]{\draw[#5,thick,rounded corners] (#1) rectangle (#2);\node at (#4) [fill=#6,thick,shape=circle,draw=#7,inner sep=2pt,font=\sffamily,text=#8] {\textbf{#3}};}
%\annotatedFigureBox{bottom-left}{top-right}{label}{label-position}
\newcommand*\annotatedFigureBox[4]{\annotatedFigureBoxCustom{#1}{#2}{#3}{#4}{white}{white}{black}{black}}
\newcommand*\annotatedFigureText[4]{\node[draw=none, anchor=south west, text=#2, inner sep=0, text width=#3\linewidth,font=\sffamily] at (#1){#4};}
\newenvironment {annotatedFigure}[1]{\centering\begin{tikzpicture}
\node[anchor=south west,inner sep=0] (image) at (0,0) { #1};\begin{scope}[x={(image.south east)},y={(image.north west)}]}{\end{scope}\end{tikzpicture}}
%%%%%%%%%%%%%%%%%%%%%%%%%%%%%%%%%%%%%%%%%%%%%%%%%%%%%%%%%%%%%%%%%%%%%%

% Packages and scripts for zooms in images
\usepackage{graphicx}
\usepackage{caption}
\usepackage{subcaption}
\usepackage{tikz}
\usepackage{pgfplots}
\usetikzlibrary{spy,calc}
\usepackage{hyperref}
\usepackage[export]{adjustbox}


\newif\ifblackandwhitecycle
\gdef\patternnumber{0}

\pgfkeys{/tikz/.cd,
    zoombox paths/.style={
        draw=orange,
        very thick
    },
    black and white/.is choice,
    black and white/.default=static,
    black and white/static/.style={ 
        draw=white,   
        zoombox paths/.append style={
            draw=white,
            postaction={
                draw=black,
                loosely dashed
            }
        }
    },
    black and white/static/.code={
        \gdef\patternnumber{1}
    },
    black and white/cycle/.code={
        \blackandwhitecycletrue
        \gdef\patternnumber{1}
    },
    black and white pattern/.is choice,
    black and white pattern/0/.style={},
    black and white pattern/1/.style={    
            draw=white,
            postaction={
                draw=black,
                dash pattern=on 2pt off 2pt
            }
    },
    black and white pattern/2/.style={    
            draw=white,
            postaction={
                draw=black,
                dash pattern=on 4pt off 4pt
            }
    },
    black and white pattern/3/.style={    
            draw=white,
            postaction={
                draw=black,
                dash pattern=on 4pt off 4pt on 1pt off 4pt
            }
    },
    black and white pattern/4/.style={    
            draw=white,
            postaction={
                draw=black,
                dash pattern=on 4pt off 2pt on 2 pt off 2pt on 2 pt off 2pt
            }
    },
    zoomboxarray inner gap/.initial=5pt,
    zoomboxarray columns/.initial=2,
    zoomboxarray rows/.initial=2,
    subfigurename/.initial={},
    figurename/.initial={zoombox},
    zoomboxarray/.style={
        execute at begin picture={
            \begin{scope}[
                spy using outlines={%
                    zoombox paths,
                    width=\imagewidth / \pgfkeysvalueof{/tikz/zoomboxarray columns} - (\pgfkeysvalueof{/tikz/zoomboxarray columns} - 1) / \pgfkeysvalueof{/tikz/zoomboxarray columns} * \pgfkeysvalueof{/tikz/zoomboxarray inner gap} -\pgflinewidth,
                    height=\imageheight / \pgfkeysvalueof{/tikz/zoomboxarray rows} - (\pgfkeysvalueof{/tikz/zoomboxarray rows} - 1) / \pgfkeysvalueof{/tikz/zoomboxarray rows} * \pgfkeysvalueof{/tikz/zoomboxarray inner gap}-\pgflinewidth,
                    magnification=3,
                    every spy on node/.style={
                        zoombox paths
                    },
                    every spy in node/.style={
                        zoombox paths
                    }
                }
            ]
        },
        execute at end picture={
            \end{scope}
            \node at (image.north) [anchor=north,inner sep=0pt] {\subcaptionbox{\label{\pgfkeysvalueof{/tikz/figurename}-image}}{\phantomimage}};
            \node at (zoomboxes container.north) [anchor=north,inner sep=0pt] {\subcaptionbox{\label{\pgfkeysvalueof{/tikz/figurename}-zoom}}{\phantomimage}};
     \gdef\patternnumber{0}
        },
        spymargin/.initial=0.5em,
        zoomboxes xshift/.initial=1,
        zoomboxes right/.code=\pgfkeys{/tikz/zoomboxes xshift=1},
        zoomboxes left/.code=\pgfkeys{/tikz/zoomboxes xshift=-1},
        zoomboxes yshift/.initial=0,
        zoomboxes above/.code={
            \pgfkeys{/tikz/zoomboxes yshift=1},
            \pgfkeys{/tikz/zoomboxes xshift=0}
        },
        zoomboxes below/.code={
            \pgfkeys{/tikz/zoomboxes yshift=-1},
            \pgfkeys{/tikz/zoomboxes xshift=0}
        },
        caption margin/.initial=4ex,
    },
    adjust caption spacing/.code={},
    image container/.style={
        inner sep=0pt,
        at=(image.north),
        anchor=north,
        adjust caption spacing
    },
    zoomboxes container/.style={
        inner sep=0pt,
        at=(image.north),
        anchor=north,
        name=zoomboxes container,
        xshift=\pgfkeysvalueof{/tikz/zoomboxes xshift}*(\imagewidth+\pgfkeysvalueof{/tikz/spymargin}),
        yshift=\pgfkeysvalueof{/tikz/zoomboxes yshift}*(\imageheight+\pgfkeysvalueof{/tikz/spymargin}+\pgfkeysvalueof{/tikz/caption margin}),
        adjust caption spacing
    },
    calculate dimensions/.code={
        \pgfpointdiff{\pgfpointanchor{image}{south west} }{\pgfpointanchor{image}{north east} }
        \pgfgetlastxy{\imagewidth}{\imageheight}
        \global\let\imagewidth=\imagewidth
        \global\let\imageheight=\imageheight
        \gdef\columncount{1}
        \gdef\rowcount{1}
        \gdef\zoomboxcount{1}
    },
    image node/.style={
        inner sep=0pt,
        name=image,
        anchor=south west,
        append after command={
            [calculate dimensions]
            node [image container,subfigurename=\pgfkeysvalueof{/tikz/figurename}-image] {\phantomimage}
            node [zoomboxes container,subfigurename=\pgfkeysvalueof{/tikz/figurename}-zoom] {\phantomimage}
        }
    },
    color code/.style={
        zoombox paths/.append style={draw=#1}
    },
    connect zoomboxes/.style={
    spy connection path={\draw[draw=none,zoombox paths] (tikzspyonnode) -- (tikzspyinnode);}
    },
    help grid code/.code={
        \begin{scope}[
                x={(image.south east)},
                y={(image.north west)},
                font=\footnotesize,
                help lines,
                overlay
            ]
            \foreach \x in {0,1,...,9} { 
                \draw(\x/10,0) -- (\x/10,1);
                \node [anchor=north] at (\x/10,0) {0.\x};
            }
            \foreach \y in {0,1,...,9} {
                \draw(0,\y/10) -- (1,\y/10);                        \node [anchor=east] at (0,\y/10) {0.\y};
            }
        \end{scope}    
    },
    help grid/.style={
        append after command={
            [help grid code]
        }
    },
}

\newcommand\phantomimage{%
    \phantom{%
        \rule{\imagewidth}{\imageheight}%
    }%
}
\newcommand\zoombox[2][]{
    \begin{scope}[zoombox paths]
        \pgfmathsetmacro\xpos{
            (\columncount-1)*(\imagewidth / \pgfkeysvalueof{/tikz/zoomboxarray columns} + \pgfkeysvalueof{/tikz/zoomboxarray inner gap} / \pgfkeysvalueof{/tikz/zoomboxarray columns} ) + \pgflinewidth
        }
        \pgfmathsetmacro\ypos{
            (\rowcount-1)*( \imageheight / \pgfkeysvalueof{/tikz/zoomboxarray rows} + \pgfkeysvalueof{/tikz/zoomboxarray inner gap} / \pgfkeysvalueof{/tikz/zoomboxarray rows} ) + 0.5*\pgflinewidth
        }
        \edef\dospy{\noexpand\spy [
            #1,
            zoombox paths/.append style={
                black and white pattern=\patternnumber
            },
            every spy on node/.append style={#1},
            x=\imagewidth,
            y=\imageheight
        ] on (#2) in node [anchor=north west] at ($(zoomboxes container.north west)+(\xpos pt,-\ypos pt)$);}
        \dospy
        \pgfmathtruncatemacro\pgfmathresult{ifthenelse(\columncount==\pgfkeysvalueof{/tikz/zoomboxarray columns},\rowcount+1,\rowcount)}
        \global\let\rowcount=\pgfmathresult
        \pgfmathtruncatemacro\pgfmathresult{ifthenelse(\columncount==\pgfkeysvalueof{/tikz/zoomboxarray columns},1,\columncount+1)}
        \global\let\columncount=\pgfmathresult
        \ifblackandwhitecycle
            \pgfmathtruncatemacro{\newpatternnumber}{\patternnumber+1}
            \global\edef\patternnumber{\newpatternnumber}
        \fi
    \end{scope}
}